\section{Brute-force metoden}

Vi kender kraften for�rsaget af tyngdekraften\footnote{Massetiltr�kningen} som er $F_g = G \frac{m_1 m_2}{r^2}$, og vi har ogs� at Newtons anden lov siger at $F = m \vec{a}$. For en partikel $i$ i et system med $N$-partikler har accelerationen:
\begin{align}
m_i \vec{a_i} &= G\sum_{j=1}\dfrac{m_i m_j}{r_{i,j}^2}(\vec{P_j}-\vec{P_i}) \Rightarrow \\
\vec{a_i} &= \sum_{j=1}\dfrac{m_j}{r_{i,j}^2}(\vec{P_j}-\vec{P_i}) 
\end{align}

Vi har ikke konstant acceleration, da vi ikke har et system i bev�gelse, $\sum F \neq 0$. Hvis vi betragter �jebliksbilledet $dt$, s� har partiklerne tiln�rmelsesvis konstant acceleration. Det udnytter vi og f�r f�lgende hhv. hastighed og position:
\[\vec{v_i}=\vec{v_{i-1}} + \vec{a_i} \cdot dt \]
\[\vec{P_i}=\vec{P_{i-1}} + \vec{v_{i-1}} \cdot dt\]
